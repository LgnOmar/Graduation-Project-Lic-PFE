\documentclass[12pt,a4paper]{article}
\usepackage[utf8]{inputenc}
\usepackage[french]{babel}
\usepackage{graphicx}
\usepackage{amsmath}
\usepackage{listings}
\usepackage{xcolor}
\usepackage{hyperref}
\usepackage{float}

% Configuration des listings pour le code
\lstset{
    language=Python,
    basicstyle=\ttfamily\small,
    keywordstyle=\color{blue},
    stringstyle=\color{red},
    commentstyle=\color{green!60!black},
    numbers=left,
    numberstyle=\tiny,
    frame=single,
    breaklines=true
}

\title{Système de Recommandation pour JibJob\\
\large Une Approche basée sur les Graphes Neuronaux Hétérogènes}
\author{Rapport Technique}
\date{\today}

\begin{document}

\maketitle

\begin{abstract}
Ce rapport présente une implémentation détaillée d'un système de recommandation d'emplois pour la plateforme JibJob, utilisant une approche moderne basée sur les réseaux de neurones à graphes hétérogènes (HGNN). Le système combine l'analyse de sentiments BERT, les embeddings de contenu d'emploi, et un réseau neuronal convolutif sur graphe (GCN) pour générer des recommandations personnalisées.
\end{abstract}

\tableofcontents

\section{Introduction}
Le système de recommandation JibJob est conçu pour fournir des recommandations d'emplois personnalisées aux utilisateurs algériens. Il exploite les dernières avancées en matière d'apprentissage profond et de traitement du langage naturel pour comprendre les préférences des utilisateurs et le contenu des offres d'emploi.

\section{Architecture du Système}
\subsection{Vue d'ensemble}
L'architecture du système se compose de plusieurs composants principaux:
\begin{itemize}
    \item Module d'analyse de sentiments basé sur BERT
    \item Pipeline de génération d'embeddings pour les offres d'emploi
    \item Réseau neuronal convolutif sur graphe hétérogène (HGCN)
    \item API FastAPI pour le service de recommandations
\end{itemize}

\subsection{Composants Principaux}
\subsubsection{Module d'Analyse de Sentiments}
Le module d'analyse de sentiments utilise BERT pour évaluer les interactions utilisateur-emploi. Il transforme les retours qualitatifs en scores quantitatifs qui enrichissent le graphe d'interaction.

\section{Pipeline de Données}
\subsection{Simulation de Données}
Le système utilise un générateur de données synthétiques pour simuler:
\begin{itemize}
    \item Profils utilisateurs
    \item Offres d'emploi
    \item Interactions utilisateur-emploi
\end{itemize}

\subsection{Prétraitement des Données}
Le pipeline de prétraitement comprend:
\begin{itemize}
    \item Génération d'embeddings BERT pour les descriptions d'emploi
    \item Normalisation des caractéristiques
    \item Construction du graphe hétérogène
\end{itemize}

\section{Modèle GCN}
\subsection{Architecture du Modèle}
Le modèle HGCN comprend:
\begin{itemize}
    \item Couches de convolution de graphe hétérogène
    \item Couches de normalisation par lots
    \item MLP pour la prédiction de liens
\end{itemize}

\begin{lstlisting}[caption=Architecture du Modèle GCN]
class HeteroGCNLinkPredictor(torch.nn.Module):
    def __init__(self, hidden_channels, num_layers, data):
        self.convs = HeteroConv({
            ('user', 'interacts_with', 'job'): SAGEConv(...),
            ('job', 'rev_interacts_with', 'user'): SAGEConv(...)
        })
\end{lstlisting}

\subsection{Processus d'Apprentissage}
L'entraînement du modèle inclut:
\begin{itemize}
    \item Division des données en ensembles d'entraînement/validation/test
    \item Optimisation avec Adam
    \item Early stopping
    \item Validation croisée
\end{itemize}

\section{API de Service}
\subsection{Points d'Entrée}
L'API expose les endpoints suivants:
\begin{itemize}
    \item \texttt{/recommendations/\{user\_id\}}: Obtenir des recommandations pour un utilisateur
    \item Paramètres de personnalisation: nombre de recommandations, filtrage
\end{itemize}

\section{Métriques de Performance}
Le système est évalué sur plusieurs métriques:
\begin{itemize}
    \item AUC-ROC
    \item Précision
    \item Rappel
    \item NDCG (Normalized Discounted Cumulative Gain)
\end{itemize}

\section{Optimisations}
\subsection{Optimisations de Performance}
Le système implémente plusieurs optimisations:
\begin{itemize}
    \item Mise en cache des embeddings
    \item Traitement par lots des prédictions
    \item Parallélisation du prétraitement
\end{itemize}

\subsection{Stratégies de Mise à l'Échelle}
Pour gérer la croissance:
\begin{itemize}
    \item Mise à jour incrémentale du graphe
    \item Mise en cache distribuée
    \item Partitionnement du graphe
\end{itemize}

\section{Tests et Validation}
\subsection{Tests Unitaires}
Les tests couvrent:
\begin{itemize}
    \item Fonctionnalités du modèle GCN
    \item Pipeline de prétraitement
    \item API de service
\end{itemize}

\subsection{Tests d'Intégration}
Validation de:
\begin{itemize}
    \item Flux de données complet
    \item Performance du système
    \item Gestion des erreurs
\end{itemize}

\section{Conclusion}
Le système de recommandation JibJob démontre l'efficacité des graphes neuronaux hétérogènes pour la recommandation d'emplois. Les résultats montrent une amélioration significative par rapport aux approches traditionnelles.

\section{Perspectives Futures}
Améliorations potentielles:
\begin{itemize}
    \item Intégration de données contextuelles supplémentaires
    \item Adaptation dynamique des recommandations
    \item Extension à d'autres types d'interactions
\end{itemize}

\appendix
\section{Détails d'Implémentation}
\subsection{Configuration du Système}
\begin{itemize}
    \item Python 3.12
    \item PyTorch Geometric
    \item FastAPI
    \item BERT (Transformers)
\end{itemize}

\subsection{Structure du Projet}
\begin{verbatim}
jibjob_recommendation/
├── data/
├── models/
├── src/
│   ├── api.py
│   ├── data_simulation.py
│   ├── feature_engineering.py
│   ├── gcn_model.py
│   ├── graph_construction.py
│   ├── pipeline.py
│   ├── recommender.py
│   └── sentiment_analysis_module.py
\end{verbatim}

\end{document}
